\documentclass{exercss}

\usepackage[ngerman]{babel}
\usepackage{dapzwei}

\usetikzlibrary{chains}

\exerciseMeta{
	sheet number=8,
	%-----------------------------------------------------------
	% ANPASSEN!
	author=Anna Apfel, % EURE NAMEN (auf dem Deckblatt)...
	author=Bert Birne, % ... und so weiter
	authors short={Apfel, Birne}, % für die Kopfzeilen
	%-----------------------------------------------------------
}

\begin{document}
	\stepcounter{exercise}
	\begin{exercise}[title={Effiziente Algorithmen}]
		\begin{subexercises}
		\item % Teilaufgabe a)
			\Todo{Verfassen!}
		\item % Teilaufgabe b)
			\Todo{Verfassen!}
		\item % Teilaufgabe c)
			\Todo{Verfassen!}
		\item % Teilaufgabe d)
			\Todo{Verfassen!}
		\end{subexercises}
	\end{exercise}

	Beispiel-Diagramm:
	
	\begin{tikzpicture}
		\tikzset{
			node distance=0.3cm,
			auto,
			block/.style={
				rectangle,
				draw,
				minimum width=3em,
			},
			math/.style={
				execute at begin node=$,
				execute at end node=$,
			},
			rectangle connector/.style={
				to path={
					(\tikztostart) --
					++(#1,0pt) \tikztonodes |- (\tikztotarget)
				},
				pos=0.5,
			}
		}

		% ABWÄRTS:
		\begin{scope}
			[
				start chain=going below,
				every on chain/.style={join=by ->},
				every node/.append style={on chain,block,math},
			]
			\node (f9) {f(9)};
			\node (f8) {f(8)};
			\node (f4) {f(4)};
			\node (f2) {f(2)};
			\node (f1) {f(1)};
		\end{scope}

		% KNOTEN DANEBEN:
		\begin{scope}[every node/.style={block,math}]
			\node (f3) [right=1cm of f4] {f(4)};
		\end{scope}

		% RECHTECKIGE KANTEN:
		\draw [->] (f9) -| (f3);
		\draw [rectangle connector=1.25cm,->] (f8) to (f2.10);
		\draw [rectangle connector,->] (f4) to (f2.20);
		\draw [->] (f3) |- (f2);
		\draw [rectangle connector, ->] (f3) to (f1.350);
		\draw [rectangle connector,->] (f2.340) to (f1.10);
	\end{tikzpicture}
\end{document}
